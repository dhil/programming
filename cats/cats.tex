\documentclass[10pt,twoside,a4paper]{article}

%%
%% Packages
%%
\usepackage[utf8]{inputenc}            % Enable UTF-8 encoding
\usepackage[T1]{fontenc}               % 8-bit output encoding
\usepackage{amsthm,amsmath,amssymb} % Maths typesetting
\usepackage{mathpartir}                % Inference rules
\usepackage{mathwidth}                 % Render character sequences nicely in math mode
\usepackage{stmaryrd}                  % semantic brackets
\usepackage{xspace}                    % proper spacing for macros in text
\usepackage{hyperref}                  % Interactive PDF
\usepackage[scaled=0.85]{beramono}     % Typewriter font
\usepackage{times}
\usepackage[square,numbers]{natbib}

%%
%% Definitions, Lemmas, Theorems...
%%
\newcounter{theorem}
\theoremstyle{plain}
\newtheorem{thm}{Theorem}[section]
\newtheorem{lemma}[thm]{Lemma}

\theoremstyle{definition}
\newtheorem{definition}[thm]{Definition}
\newtheorem{example}{Example}[section]

%%
%% Environments
%%
% array stuff
\newcommand{\ba}{\begin{array}}
\newcommand{\ea}{\end{array}}

\newcommand{\bl}{\ba[t]{@{}l@{}}}
\newcommand{\el}{\ea}

% Syntax environment
\newenvironment{eqs}{\small\ba[t]{@{}r@{~}c@{~}l@{}}}{\ea\normalsize}
\newenvironment{equations}{\normalsize\[\ba{@{}r@{~}c@{~}l@{}}}{\ea\]\normalsize\ignorespacesafterend}

%%
%% Some hacks
%%
\renewcommand{\refname}{Bibliography}  % Change bibliography section heading

%%
%% Some convenient macros
%%
\newcommand{\domain}[1]{%
  \mathrm{dom}(#1)%
}
\newcommand{\codomain}[1]{%
  \mathrm{cod}(#1)%
}
\newcommand{\range}[1]{%
  \mathrm{range}(#1)%
}
\newcommand{\id}[1]{%
  \ensuremath{\mathbf{1}_{#1}}%
}
\newcommand{\Obj}[1]{%
  \ensuremath{\mathrm{Obj}(#1)}%
}
\newcommand{\Arr}[1]{%
  \ensuremath{\mathrm{Arr}(#1)}%
}

\newcommand{\Sets}[0]{%
  \ensuremath{\mathbf{Sets}}%
}
%%
%% Meta information
%%
\author{Daniel Hillerström\\\small{Laboratory for Foundations of Computer Science}\\\small{School of Informatics, the University of Edinburgh}}
\title{\bfseries Category Theory Notes}
\date{\today}

%%
%% Body content
%%

\begin{document}
\maketitle
\begin{abstract}
  This note contains my scribbles as I have explored category theory.
  My endeavour in to category theory is mostly based on Awodey's
  excellent book ``Category Theory''~\cite{Awodey11}.
\end{abstract}

%
% Table of contents
%
\tableofcontents

%
% Preface
%
\section{Introduction}
\label{sec:intro}
\subsection{History}
Lets survey the history of category theory \dots

%
% Categories
%
\section{Categories}
\label{sec:cats}

Sometimes category theory is popularly described as the study of
algebra of functions. That is to say, we are more interested in how
functions interact than the collections they act on. The following
Definition~\ref{def:cat}, adapted from \citet{Awodey11}, makes this
popular description precise and introduces the notion of category.

\begin{definition}[Category]
\label{def:cat}
A category $\mathcal{C}$ is given by a collection of \emph{objects},
written \Obj{\mathcal{C}}, and a collection of \emph{arrows} (or
morphisms), written \Arr{\mathcal{C}}, between objects, subject to the
following conditions
\begin{itemize}
  \item For each arrow $f : A \to B$, there are given objects
\[
  \domain{f} = A, \qquad \codomain{f} = B
\]
called the \emph{domain} and \emph{codomain} of $f$, respectively.

  \item For each object $A$, there is an \emph{identity arrow} written
\[
  \id{A} : A \to A
\]

  \item For given arrows $f : A \to B$ and $g : B \to C$, that is
\[
  \codomain{f} = \domain{g},
\]
  there is a given arrow
\[
  g \circ f : A \to C
\]
called the \emph{composite} of $f$ and $g$.
\end{itemize}
The composition operator ($\circ$) is further subject to the following
laws
\begin{description}
  \item[Unit] For all arrows $f : A \to B$
\[
  f \circ \id{A} = \id{B} \circ f
\]

  \item[Associativity] For all arrows $f : A \to B, g : B \to C, h : C \to D$
\[
  h \circ (g \circ f) = (h \circ g) \circ f
\]
\end{description}
\end{definition}

As with most writings we are faced with a choice of terminology. The
above definition makes use of function syntax, i.e.  $f : A \to B$,
yet it mentions \emph{arrows} or \emph{morphisms} rather than
``functions'', and \emph{objects} $A$ and $B$ rather than ``sets''. As
we shall see shortly, sets form a category, where the objects are sets
and the arrows are exactly functions between those sets. Since set
theory predates the development of category theory, I believe this
particular choice of terminology is mostly due to historical reasons
as the set theorists had already claimed the terms ``function'' and
``set''.

Note the definition states that every object has an identity arrow,
but it says nothing about whether identity arrows are unique. As it
turns out, the uniqueness of identity arrows follows by the unit
axioms.
%
\begin{lemma}[Uniqueness of identity arrows]
  Let $\mathcal{C}$ be a category, then for any object
  $X \in \Obj{\mathcal{C}}$ the identity arrow
  $\id{X} \in \Arr{\mathcal{C}}$ is unique.
\end{lemma}
\begin{proof}
  Suppose there are two identity arrows $\id{X}$ and $\id{X}'$, then
  it follows by the unit axiom that
\[
  \id{X} = \id{X} \circ \id{X}' = \id{X}'.
\]
\end{proof}

Along the same lines, we can show that identity arrows on distinct
objects are themselves distinct.
\begin{lemma}[Distinctness of identity arrows]
  Let $\mathcal{C}$ be a category and $X,Y \in \Obj{\mathcal{C}}$ be
  two distinct objects, i.e. $X \neq Y$, then $\id{X}$ and $\id{Y}$
  are distinct.
\end{lemma}
\begin{proof}
  Observe that $X \neq Y$ implies
  $\domain{\id{X}} \neq \domain{\id{Y}}$ which entails
  $\id{X} \neq \id{Y}$.
\end{proof}

\subsection{Examples of concrete categories}

Let us study some examples of categories. We begin by considering
\emph{concrete categories}. For now it suffices to think of a concrete
category as a category whose objects are sets. Later, we shall make
this notion more precise.

\begin{example}[The category of sets]
  Regular sets (as from set theory) along with functions on them form
  a category which we shall call \Sets{}. To see this, let
  $\Obj{\Sets}$ be a collection of sets, and $\Arr{\Sets}$ be a
  collection of functions such that
\[
  f : A \to B, \quad \range{f} \subseteq B, \quad \text{for all } A,B \in \Obj{\Sets}, f \in \Arr{\Sets}.
\]
Furthermore define the identity arrow $\id{A} : A \to A$ as
\[
   \id{A}(a) = a, \quad \text{for all } a \in A.
\]

Arrow composition is simply ordinary function composition (a slight
notational overload as ($\circ$) is used to denote both composition
operators).

It remains to show that the unit and associativity laws hold
\begin{description}
  \item[Unit] Let $f : A \to B \in \Arr{\Sets}$, then
\[
  (f \circ \id{A})(a) = f(\id{A}(a)) = f(a) = \id{B}(f(a)) = (\id{B} \circ f)(a), \quad \forall a \in A.
\]
Hence it follows by extensionality that
$f \circ \id{A} = \id{B} \circ f$.

\item[Associativity] Let $f : A \to B, g : B \to C, h : C \to D \in \Arr{\Sets}$, then
\[
  (h \circ (g \circ f))(a) = h(g(f(a))) = ((h \circ g) \circ f)(a), \quad \forall a \in A.
\]
It follows by extensionality that
$h \circ (g \circ f) = (h \circ g) \circ f$. Thus we have shown that
\Sets{} is indeed a category.
\end{description}
\end{example}

Our definition of \Sets{} says nothing about the cardinality of the
sets involved. By restricting the involved sets to be finite, we
immediately obtain another category $\Sets_{\mathbf{fin}}$. One can
obtain many such categories by simply restricting the objects.

\subsection{Examples of abstract categories}

%
% References
%
\addcontentsline{toc}{section}{Bibliography}
\bibliographystyle{plainnat}
\bibliography{\jobname}
\end{document}

