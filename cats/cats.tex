\documentclass[10pt,twoside,a4paper]{article}

%%
%% Packages
%%
\usepackage[utf8]{inputenc}            % Enable UTF-8 encoding
\usepackage[T1]{fontenc}               % 8-bit output encoding
\usepackage{amsthm,amsmath,amssymb} % Maths typesetting
\usepackage{mathpartir}                % Inference rules
\usepackage{mathwidth}                 % Render character sequences nicely in math mode
\usepackage{stmaryrd}                  % semantic brackets
\usepackage{xspace}                    % proper spacing for macros in text
\usepackage{hyperref}                  % Interactive PDF
\usepackage[scaled=0.85]{beramono}     % Typewriter font
\usepackage{times}
\usepackage[square,numbers]{natbib}

%%
%% Definitions, Lemmas, Theorems...
%%
\newcounter{theorem}
\theoremstyle{plain}
\newtheorem{thm}{Theorem}[section]
\newtheorem{lemma}[thm]{Lemma}

\theoremstyle{definition}
\newtheorem{definition}[thm]{Definition}
\newtheorem{example}{Example}[section]

%%
%% Environments
%%
% array stuff
\newcommand{\ba}{\begin{array}}
\newcommand{\ea}{\end{array}}

\newcommand{\bl}{\ba[t]{@{}l@{}}}
\newcommand{\el}{\ea}

\newenvironment{eqs}{\small\ba[t]{@{}r@{~}c@{~}l@{}}}{\ea\normalsize}
\newenvironment{equations}{\normalsize\[\ba{@{}r@{~}c@{~}l@{}}}{\ea\]\normalsize\ignorespacesafterend}
\newenvironment{derivation}{\small\[\ba{@{}r@{~}l@{}}}{\ea\]\normalsize\ignorespacesafterend}
\newcommand{\reason}[1]{\quad (\text{#1})}

%%
%% Some hacks
%%
\renewcommand{\refname}{Bibliography}  % Change bibliography section heading

%%
%% Some convenient macros
%%
\newcommand{\Domain}[1]{%
  \mathrm{dom}(#1)%
}
\newcommand{\Codomain}[1]{%
  \mathrm{cod}(#1)%
}
\newcommand{\Range}[1]{%
  \mathrm{range}(#1)%
}
\newcommand{\Id}[1]{%
  \ensuremath{\mathbf{1}_{#1}}%
}
\newcommand{\Obj}[1]{%
  \ensuremath{\mathrm{Obj}(#1)}%
}
\newcommand{\Arr}[1]{%
  \ensuremath{\mathrm{Arr}(#1)}%
}

\newcommand{\Catname}[1]{%
  \ensuremath{\mathbf{#1}}%
}

\newcommand{\Sets}[0]{\Catname{Sets}}
\newcommand{\Pos}[0]{\Catname{Pos}}
\newcommand{\Mon}[0]{\Catname{Mon}}

\newcommand{\Tuple}[1]{%
  \ensuremath{\langle #1 \rangle}%
}

%%
%% Meta information
%%
\author{Daniel Hillerström\\\small{Laboratory for Foundations of Computer Science}\\\small{School of Informatics, the University of Edinburgh}}
\title{\bfseries Category Theory Notes}
\date{\today}

%%
%% Body content
%%

\begin{document}
\maketitle
\begin{abstract}
  This note exists solely for me to compose my thoughts and deepen my
  understanding during my endeavour of category theory.  The goal of
  this note is to develop the necessary artillery to study Lawvere
  theories.
\end{abstract}

%
% Table of contents
%
\setcounter{tocdepth}{2}
\tableofcontents

%
% Preface
%
\section{Introduction}
\label{sec:intro}
\subsection{History}
Lets survey the history of category theory \dots

%
% Categories
%
\section{Categories}
\label{sec:cats}

Sometimes category theory is popularly described as the study of
algebra of functions. That is to say, we are more interested in how
functions interact than the collections they act on. The following
Definition~\ref{def:cat}, adapted from \citet{Awodey11}, makes this
popular description precise and introduces the notion of category.

\begin{definition}[Category]
\label{def:cat}
A category $\mathcal{C}$ is given by a collection of \emph{objects},
written \Obj{\mathcal{C}}, and a collection of \emph{arrows} (or
morphisms), written \Arr{\mathcal{C}}, between objects, subject to the
following conditions
\begin{itemize}
  \item For each arrow $f : A \to B$, there are given objects
\[
  \Domain{f} = A, \qquad \Codomain{f} = B
\]
called the \emph{domain} and \emph{codomain} of $f$, respectively.

  \item For each object $A$, there is an \emph{identity arrow} written
\[
  \Id{A} : A \to A
\]

  \item For given arrows $f : A \to B$ and $g : B \to C$, that is
\[
  \Codomain{f} = \Domain{g},
\]
  there is a given arrow
\[
  g \circ f : A \to C
\]
called the \emph{composite} of $f$ and $g$.
\end{itemize}
The composition operator ($\circ$) is further subject to the following
laws
\begin{description}
  \item[Unit] For all arrows $f : A \to B$
\[
  f \circ \Id{A} = \Id{B} \circ f
\]

  \item[Associativity] For all arrows $f : A \to B, g : B \to C, h : C \to D$
\[
  h \circ (g \circ f) = (h \circ g) \circ f
\]
\end{description}
\end{definition}

As with most writings we are faced with a choice of terminology. The
above definition makes use of function syntax, i.e.  $f : A \to B$,
yet it mentions \emph{arrows} or \emph{morphisms} rather than
``functions'', and \emph{objects} $A$ and $B$ rather than ``sets''. As
we shall see shortly, sets form a category, where the objects are sets
and the arrows are exactly functions between those sets. Since set
theory predates the development of category theory, I believe this
particular choice of terminology is mostly due to historical reasons
as the set theorists had already claimed the terms ``function'' and
``set''.

Note the definition states that every object has an identity arrow,
but it says nothing about whether identity arrows are unique. As it
turns out, the uniqueness of identity arrows follows by the unit
axioms.
%
\begin{lemma}[Uniqueness of identity arrows]
  Let $\mathcal{C}$ be a category, then for any object
  $X \in \Obj{\mathcal{C}}$ the identity arrow
  $\Id{X} \in \Arr{\mathcal{C}}$ is unique.
\end{lemma}
\begin{proof}
  Suppose there are two identity arrows $\Id{X}$ and $\Id{X}'$, then
  it follows by the unit axiom that
\[
  \Id{X} = \Id{X} \circ \Id{X}' = \Id{X}'.
\]
\end{proof}

Along the same lines, we can show that identity arrows on distinct
objects are themselves distinct.
\begin{lemma}[Distinctness of identity arrows]
  Let $\mathcal{C}$ be a category and $X,Y \in \Obj{\mathcal{C}}$ be
  two distinct objects, i.e. $X \neq Y$, then $\Id{X}$ and $\Id{Y}$
  are distinct.
\end{lemma}
\begin{proof}
  Observe that $X \neq Y$ implies
  $\Domain{\Id{X}} \neq \Domain{\Id{Y}}$ which entails
  $\Id{X} \neq \Id{Y}$.
\end{proof}

\subsection{Examples of categories}

Let us study some examples of categories. We begin by considering
categories whose objects are sets.

\subsubsection{Set-theoretic categories}

\begin{example}[Category of sets]
\label{ex:sets}
  Regular sets (as from set theory) along with functions on them form
  a category which we shall call \Sets{}. To see this, let
  $\Obj{\Sets}$ be a collection of sets, and $\Arr{\Sets}$ be a
  collection of functions such that
\[
  f : A \to B, \quad \Range{f} \subseteq B, \quad \text{for all } A,B \in \Obj{\Sets}, f \in \Arr{\Sets}.
\]
Furthermore define the identity arrow $\Id{A} : A \to A$ as
\[
   \Id{A}(a) = a, \quad \text{for all } a \in A.
\]

Arrow composition is simply ordinary function composition (a slight
notational overload as ($\circ$) is used to denote both composition
operators).

It remains to show that the unit and associativity laws hold
\begin{description}
  \item[Unit] Let $f : A \to B \in \Arr{\Sets}$, then
\[
  (f \circ \Id{A})(a) = f(\Id{A}(a)) = f(a) = \Id{B}(f(a)) = (\Id{B} \circ f)(a), \quad \forall a \in A.
\]
Hence it follows by extensionality that
$f \circ \Id{A} = \Id{B} \circ f$.

\item[Associativity] Let $f : A \to B, g : B \to C, h : C \to D \in \Arr{\Sets}$, then
\[
  (h \circ (g \circ f))(a) = h(g(f(a))) = ((h \circ g) \circ f)(a), \quad \forall a \in A.
\]
It follows by extensionality that
$h \circ (g \circ f) = (h \circ g) \circ f$. Thus we have shown that
\Sets{} is indeed a category.
\end{description}
\end{example}

Our definition of \Sets{} says nothing about the cardinality of the
sets involved. By restricting the involved sets to be finite, we
immediately obtain another category $\Sets_{\mathbf{fin}}$. One can
obtain many such categories by simply restricting the objects. In
\Sets{} we can even go as far as insisting that involved sets are
empty. In other words, $\Sets{}_\emptyset$ is a category with only one
object, namely, $\emptyset$. It is worthwhile to remind ourselves why
$\Sets{}_\emptyset$ is a well-defined category: the function
$f : \empty \to Y$ is well-defined, i.e. for all $x \in \emptyset$
there is a unique $y \in Y$, in particular by choosing $Y = \emptyset$
we obtain the identity function
$\Id{\emptyset} : \emptyset \to \emptyset$ for the sole object
$\emptyset$ in $\Sets{}_{\emptyset}$. Clearly, both the unit and
associativity laws hold.

\begin{example}[Category of posets~\cite{Awodey11}]
  Recall that a partially ordered set (poset) is a set $A$ equipped
  with a binary relation $a \sqsubseteq_A b$ such that the following:
  \begin{description}
    \item[Reflexivity] $a \sqsubseteq_A a$.
    \item[Antisymmetry] if $a \sqsubseteq_A b$ and $b \sqsubseteq_A a$, then $a = b$.
    \item[Transitivity] if $a \sqsubseteq_A b$ and $b \sqsubseteq_A c$, then $a \sqsubseteq_A c$.
  \end{description}
  A function $f : A \to B$ on ordered sets is said to be \emph{monotone} if
\[
  a \sqsubseteq_A a' \quad \text{implies} \quad f(a) \sqsubseteq_B f(a'),
\]
that is the function preserves the relative ordering amongst elements.

Posets along with monotone functions form a category which we shall
call \Pos{}. The construction of \Pos{} is straightforward: let
\Obj{\Pos} be a collection of posets and \Arr{\Pos} be a collection of
monotone functions between those posets. Define the identity arrow to
be the identity function as in Example~\ref{ex:sets}. The identity
function is trivially monotone. Furthermore, monotone functions are
closed under composition since for all
$f : A \to B,g : B \to C \in \Arr{\Pos}$ and $a,a' \in A$ we have
\begin{derivation}
 & a \sqsubseteq_A a' \\
 \Rightarrow & \reason{monotonicity of $f$} \\
 &  f(a) \sqsubseteq_B f(a') \\
 \Rightarrow & \reason{monotonicity of $g$} \\
 &   g(f(a)) \sqsubseteq_C g(f(a')).
\end{derivation}
It follows that $g \circ f$ is monotone, and therefore
$g \circ f \in \Arr{\Pos}$. The proofs of the unit and associativity
laws are the same as in Example~\ref{ex:sets}.
\end{example}

\begin{example}[Category of monoids]
A monoid $M$ is given by a triple
\[ M = \Tuple{A,e \in A, \otimes : A \times A \to A}, \]
where $A$ is the carrier set, $e$ is a distinguished element in $A$,
and $\otimes$ is a binary operation such that
\begin{description}
\item[Unit] The distinguished element $e \in M$ is an identity
  element, i.e.
  \[ e \otimes a = a = a \otimes e, \quad \forall a \in M. \]
  (We abuse notation a bit and write $a \in M$ to mean $a \in A_M$,
  where $M$ is a monoid with carrier set $A_M$.)
  \item[Associativity] The binary operator is associative, i.e.
    \[  a \otimes (b \otimes c) = (a \otimes b) \otimes c, \quad \forall a,b,c \in M. \]
\end{description}
The definition of monoid is reminiscent of the definition of
category. In fact, every monoid $M$ induce a category \Catname{M} with
a single object which is the carrier set of $M$. The arrows are the
elements of $M$, that is $\Arr{\Catname{M}} = \{ a \in A \}$. Thus
arrows in this category are not functions. The identity arrow is the
distinguished element of $M$. Arrow composition is defined as follows
%
\[ g \circ f = f \otimes g, \quad \forall f,g \in \Arr{\Catname{M}}. \]
%
The category associativity and unit laws hold trivially by the
definition of the binary operator $\otimes$.

There is also the category \Mon{} whose objects are monoids. In this
category arrows are monoid homomorphisms. Recall that a function
$f : M \to N$ between two monoids $M$ and $N$ is said to be
\emph{homomorphism} if $f(e) = e'$ that is it maps the distinguished
element $e \in M$ to the distinguished element $e' \in N$, and
%
\[ f(a \otimes_M a') = f(a) \otimes_N f(a'), \quad \forall a,a' \in M. \]
%
The identity arrow is the identity function $\Id{M} : M \to M$ which
is trivially a homomorphism. Suppose $f : M_1 \to M_2$ and
$g : M_2 \to M_3$ are homomorphisms, then their composition is also a
homomorphism since for all $a, a' \in M$
\begin{derivation}
  & (g \circ f)(a \otimes_{M_1} a') \\
  =& \reason{definition} \\
  & g(f(a \otimes_{M_1} a')) \\
  =& \reason{$f$ is a homomorphism} \\
  & g(f(a) \otimes_{M_2} f(a')) \\
  =& \reason{$g$ is a homomorphism} \\
  & g(f(a)) \otimes_{M_3} g(f(a')).
\end{derivation}
Moreover, homomorphism composition is associative since ordinary
function composition is associative. Thus it follows that \Mon{} is a
category.
\end{example}

%
% References
%
\addcontentsline{toc}{section}{Bibliography}
\bibliographystyle{plainnat}
\bibliography{\jobname}
\end{document}

