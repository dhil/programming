\documentclass[a4paper,11pt]{article}
\usepackage[utf8]{inputenc}   % Enable UTF-8 tying
\usepackage[T1]{fontenc}      % Fixes issues with accented characters
\usepackage[british]{babel}   % British English
\usepackage{amsmath}          % Mathematics library
\usepackage{amssymb}          % Provides math fonts
\usepackage{amsthm}           % Provides \newtheorem, \theoremstyle, etc.
\usepackage{mathtools}
% Theorem environments
\theoremstyle{plain}
\newtheorem{theorem}{Theorem}[section]
\newtheorem{lemma}[theorem]{Lemma}
\newtheorem{proposition}[theorem]{Proposition}
\newtheorem{corollary}[theorem]{Corollary}
\theoremstyle{definition}
\newtheorem{definition}[theorem]{Definition}
% Example environment.
\makeatletter
\def\@endtheorem{\hfill$\blacksquare$\endtrivlist\@endpefalse} % inserts a black square at the end.
\makeatother
\newtheorem{example}{Example}[section]
% Lindley's equations and derivation environments.
\newcommand{\ba}{\begin{array}}
\newcommand{\ea}{\end{array}}
\newcommand{\bl}{\ba[t]{@{}l@{}}}
\newcommand{\el}{\ea}
\newenvironment{eqs}{\ba{@{}r@{~}c@{~}l@{}}}{\ea}
\newenvironment{equations}{\[\ba{@{}r@{~}c@{~}l@{}}}{\ea\]\ignorespacesafterend}
\newcommand\numberthis{\addtocounter{equation}{1}\tag{$\ast$\theequation}} % Numbering equations
\newenvironment{derivation}{\begin{displaymath}\ba{@{}r@{~}l@{}}}{\ea\end{displaymath}\ignorespacesafterend}
\newcommand{\reason}[1]{\quad (\text{#1})}

% Metadata
\author{Daniel Hillerström}
\date{\today}
\title{Solved Exercises in Proof by Induction}

\begin{document}
\maketitle
\section{The Induction Principle}
The induction principle can be summarised as follows.
\begin{definition}
  Let $P(n)$ be a statement which depends on $n \in \mathbb{N}$. Then
  $P(n)$ is true for all $n$ if
  \begin{description}
    \item[Base step] $P(1)$ is true.
    \item[Inductive step] Assume $P(k)$ is true, then show that with
      this assumption $P(k + 1)$ must be true.
  \end{description}
\end{definition}

\section{Summations}

\begin{theorem}
  \[
    \displaystyle\sum_{i=1}^ni = \frac{n(n+1)}{2}
  \]
\end{theorem}
%
\begin{proof}
  The universe of discourse is $\mathbb{N} = \{1,2,3,\dots\}$ and $i,n\in\mathbb{N}$.
  \begin{description}
  \item[Base step] $n = 1$. This case follows by direct calculation.
    \[
      1 = \frac{1\cdot(1+1)}{2}
    \]
  \item[Inductive step] Assume $\sum_{i=1}^ki = \frac{k(k+1)}{2}$.
    The induction hypothesis can also be written as follows.
    %
    \[
      1+2\cdots+k = \frac{k(k+1)}{2}.
    \]
    %
    Now take $n = k + 1$. We need to show the following.
    \[
      1+2+\cdots+k+(k+1) = \frac{(k+1)((k+1)+1)}{2}.
    \]
    %
    The proof proceeds by direct calculation.
    %
    \begin{derivation}
      &1+2\cdots+k+(k+1)\\
     =& \reason{induction hypothesis}\\
      &\displaystyle\frac{k(k+1)}{2} + (k+1)\\
     =& \reason{multiply by $1 = \frac{2}{2}$}\\
      & \displaystyle\frac{k(k+1)}{2} + \frac{2(k+1)}{2}\\
     =& \reason{fraction addition}\\
      & \displaystyle\frac{k(k+1)+2(k+1)}{2}\\
     =& \reason{distribution of multiplication over addition}\\
      &\displaystyle\frac{k^2+3k+2}{2}\\
     =&\reason{factorisation}\\
      &\displaystyle\frac{(k+1)(k+2)}{2} \\
    \end{derivation}
  \end{description}
\end{proof}

\begin{theorem}
  \[
    \displaystyle\sum_{i=1}^ni^2 = \frac{n(n + 1)(2n + 1)}{6}
  \]
\end{theorem}
\begin{proof}
  The universe of discourse is $\mathbb{N} = \{1,2,3,\dots\}$ and $i,n\in\mathbb{N}$.
  \begin{description}
  \item[Base step] $n = 1$. This case follows by direct calculation.
    \[
      1^2 = \frac{1*(1 + 1)(2 * 1 + 1)}{6} = 1.
    \]
  \item[Inductive step] Assume
    $\sum_{i = 1}^ki^2 = \frac{k(k + 1)(2k + 1)}{6}$. The induction
    hypothesis can alternatively be written as follows.
    %
    \[
      \displaystyle 1+4+9+16+25+36+\cdots+k^2 = \frac{k(k+1)(2k+1)}{6}.
    \]
    %
    We need to show that $1+4+\cdots+k^2+(k + 1)^2 = \frac{(k+1)(k+2)(2k+3)}{6}$.
    %
    \begin{derivation}
      &1+4+\cdots+k^2+(k+1)^2\\
      =& \reason{induction hypothesis}\\
      &\displaystyle\frac{k(k+1)(2k+1)}{6} + (k+1)^2\\
      =& \reason{multiply by $1 = \frac{6}{6}$}\\
      &\displaystyle\frac{k(k+1)(2k+1)+6(k+1)^2}{6}\\
      =& \reason{multiplication}\\
      &\displaystyle\frac{2k^3+9k^2+13k+6}{6} = \frac{(k+1)(k+2)(2k+3)}{6}\\
    \end{derivation}
  \end{description}
\end{proof}

\begin{theorem}
  \[
    \displaystyle\sum_{i=1}^n2^{i-1} = 2^n-1
  \]
\end{theorem}
\begin{proof}
  The universe of discourse is $\mathbb{N} = \{1,2,3,\dots\}$ and $i,n\in\mathbb{N}$.
  \begin{description}
  \item[Base step] $n = 1$. This case follows by direct calculation.
    \[
      2^0 = 2^1 - 1 = 1
    \]
  \item[Inductive step] Assume $\sum_{i = 1}^k2^{i-1} = 2^k - 1$. The
    induction hypothesis can also be written as follows.
    %
    \[
      1 + 2 + 4 + 8 + 16 + \cdots + 2^{k-1} = 2^k - 1.
    \]
    %
    We need to show that $1+2+\cdots+2^{k-1}+2^k = 2^{k+1} - 1$.
    %
    \begin{derivation}
      &1+2+\cdots+2^{k-1}+2^k\\
      =& \reason{induction hypothesis}\\
      &(2^k - 1) + 2^k\\
      =& \reason{factorise common term}\\
      & 2*2^k - 1\\
      =& \reason{exponentiation}\\
      &2^{k+1} - 1
    \end{derivation}
  \end{description}
\end{proof}


\end{document}